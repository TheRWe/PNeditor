
%%%  Ukázkový text a dokumentace stylu pro text závěrečné (bakalářské a
%%%  diplomové) práce na KI PřF UP v Olomouci
%%%  Copyright (C) 2012 Martin Rotter, <rotter.martinos@gmail.com>
%%%  Copyright (C) 2014 Jan Outrata, <jan.outrata@upol.cz>

\documentclass[
  biblatex,
  glossaries,
  printversion
]{kidiplom}

\usepackage{algorithm,algorithmicx,algpseudocode}


\title{Editor Petriho Sítí}
\title[english]{Petri Nets Editor}


\iffalse
\subtitle{Ukázkový text a dokumentace stylu v \LaTeX{}u}
\subtitle[english]{Sample text and documentation of the \LaTeX{} style}
\fi

\author{Roman Wehmhőner}

\supervisor{Mgr. Petr Osička, Ph.D.}


\annotation{Cílem bakalářské práce bylo vytvořit editor 
  Petriho sítí umožňující jednoduché a pohodlné ovládání. 
  Editor také obsahuje základní nástroje pro analýzu Petriho 
  sítí. }

\annotation[english]{The goal of bachelor thesis is to make editor for Petri 
nets with simple convenient control. Editor contains basic analysis 
tools too.}


\keywords{Petriho síť; Editor }
\keywords[english]{Petri net; Editor}


\thanks{Tímto bych chtěl poděkovat vedoucímu bakalářské práce 
panu Mgr. Petrovi Osičkovi, Ph.D. za hodnotné rady při vývoji editoru.
}

\bibliography{bibliografie.bib}

\usepackage{lipsum}



\usepackage{enumitem}
\newlist{steps}{enumerate}{1}
\setlist[steps, 1]{label = Krok \arabic*:}

\usepackage{graphicx}
\graphicspath{ {./images/} }

\usepackage{subcaption}
\usepackage[czech]{babel}

\usepackage{multirow}
\usepackage{array}

\begin{document}





\maketitle

\newcommand{\BibLaTeX}{\textsc{Bib}\LaTeX}
\newcommand{\todo}[1]{\textcolor{red}{TODO: #1}\PackageWarning{TODO:}{#1!}}



\section{Petriho sítě}
Tato kapitola byla inspirovaná a čerpala informace z knihy 
Understanding petri nets\cite{reisig2013understanding}.


\subsection{Základní popis}

Petriho síťe jsou matematickým nástrojem
pro modelování a simulaci paralelních procesů a jejich sychronizaci.
Jsou tvořené místy, přechody a hranami. Každá hrana vždy propojuje
jeden přechod s jedním místem.

\begin{definition}[Petriho síť]
  $$ N = \langle P,T,A,M_{0}\rangle $$
  \begin{itemize}
    \item $N$ je Petriho sítí.
    \item $P$ je konečná množina míst.
    \item $T$ je konečná množina přechodů.
    \item $A$ je konečná množina hran.
        $ A \subseteq ((P \times T) \cup (T \times P)) \times \mathbb{N}_0 $ \\
        kde číslo symbolizuje násobek kolik značek hrana \uv{přesune}.
    \item $M_{0} : P \rightarrow \mathbb{N}_0$ je počáteční ohodnocení míst sítě (zkráceně ohodnocení),
        kde pro každé místo $p \in P$ existuje počet jeho značek $M_{0}(p) = \mathbb{N}_0$.
  \end{itemize}
\end{definition}

Pro odkazovaní na jednotlivé členy prvků z množiny hran $a \in A$ 
budeme používat notaci $P(a)$ pro odkázání na místo hrany $a$,
$T(a)$ pro odkázání na přechod a $AM(a)$ pro odkaz na násobek.

Každý přechod $t$ může mít \uv{přiřazený} libovolný počet
hran $a \in A_t$, kde každá hrana $a$ je propojením přechodu $t$ s některým z míst $p \in P$.\\
Hrany přechodu $t$ můžeme rozlišit na hrany směrující do přechodu
$$^\to t = \{a \in A | a \in (P \times T \times \mathbb{N}_0) \land t = T(a)\}$$
a hrany směřující z přechodu (do místa)
$$ t ^\to  = \{a \in A | a \in (T \times P \times \mathbb{N}_0) \land t = T(a)\}$$
dohromady pak všechny hrany přechodu $t$ jsou spojením těchto dvou množin
$$ArcesOfTransition(t,A) = A_t = (^\to t \cup t ^\to)\;\; \subset A$$

Mezi libovolným přechodem $t \in T$ a libovolným místem $p \in P$ může existovat \textbf{maximálně jedna} 
hrana $a_{pt} \in \,^\to t$
a \textbf{maximálně jedna} hrana $a_{tp} \in t ^\to$ (dohmady tedy maximálně 2 hrany, jedna \textbf{z} a druhá \textbf{do} přechodu).

Aktuální stav Petriho sítě neboli ohodnocení M je funkce přiřazující každému
místu $p \in P$ Petriho sítě počet značek
$$(\forall p \in P) M(p) \in \mathbb{N}_0$$
Počáteční ohodnocení Petriho sítě se značí $M_{0}$.

Pro libovolné ohodnocení $M$ je přechod $t \in T$ označený jako \textbf{povolený},
pokud všechny hrany směřující do přechodu $\,^\to t$ splňují svou podmínku tzn.
hrana splňuje svoji podmínku, pokud místo ze kterého vychází má vyšší nebo stejné ohodnocení
(v daném $M$) než je násobek hrany $AM$
$$IsEnabled(P,t,A,M) = (\forall a \in \,^\to t)M(P(a)) \geq AM(a)$$
Pak si můžeme ještě definovat množinu všech povolených přechodů pro zadané ohodnocení
$$
  EnabledTransitions(P,T,A,M) =
  \{ t \in T | IsEnabled(P,t,A,M) \}
$$

Pokud je přechod $t$ v ohodnocení $M$ Petriho sítě \textbf{povolený}, znamená to že může dojít
k \textbf{aktivování} tohoto přechodu, čímž dojde ke změně aktuálního ohodnocení z $M$ do ohodnocení $M'$
tak, že pro každé místo $p \in P$ a hrany $A_{t} \subset ArcesOfTransition(t,A)$ 
spojující $p$ s $t$ že nové ohodnocení 
v místě $M'(p)$ je sumou násobků 
hran $\sum_{a \in A_{t}} AM(a)$ a původního hodnocení $M(p)$
$$
  FireTransition(P,t,A,M) = function\:M'
$$
Výsledné ohodnocení $M'$ je pak pro každé místo $p$ definováno
$$
  M'(p) = M(p) + \sum_{a \in \{a_{tp} \in ArcesOfTransition(t,A) \,|\, P(a_{tp}) = p \}} AM(a)
$$
Tuto změnu značíme $M \to ^t M'$.

Ohodnocení $M'$ je označené jako 
\textbf{dosažitelné} z ohodnocení $M$, 
pokud existuje sekvence přechodů taková, že jejich postupným 
\textbf{aktivováním} z ohodnocení $M$ vznikne ohodnocení $M'$.
Ohodnocení $M'$ je dostupné z ohodnocení $M$ pak značíme 
$M \to ^* M'$.


\subsection{Vizuální zobrazení sítě}

Pro Petriho síť existuje nejenom matematické zobrazení, ale i
v praxi více využívané grafické zobrazení. 
V grafickém zobrazení kolečka symbolizují místa Petriho sítě
a číslo v kolečku udává počet značek. 
Přechody jsou symbolizované 
čtverci. V editoru je čtverec zelený, pokud je přechod povolený.
Pokud má v sobě čtverec symbol $\epsilon$, znamená to že je 
přechod určený pro komunikaci sítě s okolním prostředím.
Na editor nemá žádný vliv jestli je přechod označený jako $\epsilon$
a tak tedy toto označení je jen pro jednodušší orientaci v síti.
Kolečka i čtverce mají pak nad sebou značení 
konkrétního místa/přechodu které symbolizují. 
Nakonec samotné hrany jsou symbolizované šipkami které jsou 
popsané násobkem kolik hrana \uv{přesune} značek.
Popis hran je symbolizován dvěmi čísly oddělenými kolečkem.
Číslo před kolečkem značí hodnotu hrany směrující z přechodu 
do místa, číslo za kolečkem značí hodnotu hrany směrující z 
místa do přechodu.

Na obrázku \ref{fig:jednoduchá síť zobrazení} můžeme vidět grafické 
vyobrazení jednoduché sítě s dvěmi místy, třemi přechody a čtyřmi hranami.
\begin{figure}[h]
  \centering
  \includegraphics[scale=0.8]{simple_net}
  \caption{Příklad zobrazení jednoduché sítě}\label{fig:jednoduchá síť zobrazení}
\end{figure}
\\Již na první pohled si můžeme všimnou, že pro jednoduší rozlišení 
míst a přechodů jsou přechody značené čísli a místa písmeny.
Tuto síť vyobrazenou na obrázku \ref{fig:jednoduchá síť zobrazení} 
bychom mohli matematickým zápisem zapsat jako síť 
$ N = \langle P,T,A,M_{0}\rangle $
kde \\
$P = \{a,b\}$ \\
$T = \{1,2,3\}$ \\
$A = \{
  \langle 1,a,1 \rangle,
  \langle 2,a,1 \rangle,
  \langle 2,b,1 \rangle,
  \langle b,2,2 \rangle,
  \langle a,3,1 \rangle
\}$ \\
$M_{0}(a) = 0; M_{0}(b) = 2$

V matematickém zápise sítě budeme místa značit malými písmeny (oproti editoru)
abychom předešli případným nedorozuměním.


\subsection{Zkrácený zápis ohodnocení sítě}

Abychom se vyhnuli zdlouhavému psaní každého případu ohodnocovací funkce
\\(např. $M(a) = 0; M(c) = 2; \dotsb$), zavedeme si kratší zápis.
Nejdříve seřadíme všechny místa podle jejich značení abecedně jakoby šlo 
o číselnou soustavu (s písmeny, bez čísel) 
neboli $a, b, c, \dotsb, z, aa, ab, ac, \dotsb$.
Pak si z těchto míst uděláme uspořádanou n-tici jejich ohodnocení $\langle M(a), M(b), M(c), \dotsb \rangle$.
Tuto n-tici pak budeme používat jako kratší zápis ohodnocovací funkce:
$$
 M = \langle M(a), M(b), M(c), \dotsb \rangle
$$
Například pro $M'(a) = 3; M'(b) = 0; M'(c) = 5$ je krátký zápis $M' = \langle 3,0,5 \rangle$


\subsection{Využití Petriho sítí}

Petriho sítě se používají k analýze a modelování paralelních
a distribuovaných systému, databázových systémů atd. a to ať už
pro analýzu při vývoji softwaru a nebo pro popis vnitřní struktury
již hotového proprietárního softwaru umožnující lepší porozumění uživateli.


\subsection{Graf dosažitelnosti}

Graf dosažitelnosti je jeden z nejzákladnějších nástrojů pro analýzu Petriho sítí.
Obsahuje vždy počáteční ohodnocení a všechny ohodnocení které jsou \textbf{dosažitelné} z počátečního ohodnocení, 
takovéto ohodnocení budeme zkráceně nazývat \textbf{dosažitelné ohodnocení}. 
Vrcholy grafu jsou jednotlivá ohodnocení
a hrany grafu jsou značené přechody které jsou aktivované aby z počátečního ohodnocení vzniklo cílové.

\begin{definition}[Graf dosažitelnosti]
  $$RG = \{M, \langle M, T', M' \rangle\}$$
  \begin{itemize}
    \item $RG$ je Graf dosažitelnosti
    \item $M$ je Vrchol grafu který je zároveň konkrétní ohodnocení Petriho sítě
    \item $\langle M, T', M' \rangle$ je Hrana grafu která je změnou z hodnocení $M$ libovolným přechodem $t \in T'$ ze kterého vzniká $M'$
  \end{itemize}
\end{definition}

\subsubsection{Vlastnosti odvoditelné z Grafu dosažitelnosti}
Z grafu dosažitelnosti Petriho sítě jsou odvoditelné tyto vlastnosti:

\begin{definition}[Ohraničenost sítě]
  
  Petriho síť je \textbf{ohraničená}, pokud 
  je její graf dosažitelnosti konečný. Pokud existuje takové přirozené číslo $n$
  pro které v žádném dosažitelném ohodnocení nepřesahuje žádné místo svým ohodnocením 
  číslo $n$ a zvolíme $n$ aby splňovalo tuto podmínku a zároveň bylo 
  nejmenší možné, pak můžeme nazvat síť že je \textbf{ohraničená} číslem $n$.
  
\end{definition}
\begin{definition}[Konečnost sítě]\label{def:skončí}
  
  Petriho síť \textbf{skončí} za předpokladu
  že graf je konečný a zároveň neobsahuje žádné cykly.
  Neboli Petriho síť vždy po nějakém počtu kroků dojde do stavu, kdy žádný přechod není povolený.
  
\end{definition}
\begin{definition}[Vratnost sítě]
  
  Petriho síť je \textbf{vratná},
  pokud je její graf silně souvislý. Z každého dosažitelného 
  ohodnocení je dosažitelné počáteční ohodnocení.

\end{definition}
\clearpage
\begin{definition}[Síť bez mrtvého bodu]
  
  Petriho síť je \textbf{bez mrtvého bodu},
  pokud z každého vrcholu grafu
  dosažitelnosti vede minimálně jedna hrana.
  Petriho síť má v každém ohodnocení povolený minimálně jeden přechod.

\end{definition}
\begin{definition}[Slabě živá síť]
  
  Petriho síť je \textbf{slabě živá}, pokud pro
  každý přechod existuje v grafu dosažitelnosti hrana označená tímto přechodem.
  Pro každý přechod Petriho sítě existuje dosažitelné ohodnocení které 
  povoluje daný přechod.

\end{definition}
\begin{definition}[Živá síť]
  
  Petriho síť je \textbf{živá}, pokud 
  pro každý přechod $t$ a každé ohodnocení $M$ existuje v grafu dosažitelnosti cesta
  z ohodnocení $M$ do ohodnocení ze kterého vede hrana s označením přechodu $t$.
  Pro každý přechod $t$ a každé ohodnocení $M$ existuje dosažitelné ohodnocení $M'$ které přechod $t$ povoluje.
  
\end{definition}
  
Logickou úvahou a z vlastností grafů pak můžeme určit některé vzájemné
 závislosti vlastností:
\begin{itemize}
  \item Síť která je \textbf{vratná} nebo/a \textbf{živá} je zároveň i \textbf{bez mrtvého bodu}.
  \item Síť která je \textbf{bez mrtvého bodu} ne\textbf{skončí} a zároveň síť která \textbf{skončí} není \textbf{bez mrtvého bodu}(pozor, neznamená že síť musí mít alespoň jednu z těchto vlastností). 
  \item Síť která není \textbf{slabě živá} nemůže být ani \textbf{živá}.
\end{itemize}

\clearpage
\subsubsection{Příklady grafu dosažitelnosti s vlastnostmi}\label{příklady sítí}

\begin{figure}[h!]
  \centering
  \begin{subfigure}[h]{\linewidth}
    \centering
    \includegraphics{net_props_01}
    \caption{}\label{fig:ukázka grafu dosažitelnosti 1}
  \end{subfigure}
  
  \begin{subfigure}[h]{\linewidth}
    \includegraphics[width=\linewidth]{net_props_02}
    \caption{}\label{fig:ukázka grafu dosažitelnosti 2}
  \end{subfigure}
  
  \begin{subfigure}[h]{\linewidth}
    \includegraphics[width=\linewidth]{net_props_03}
    \caption{}\label{fig:ukázka grafu dosažitelnosti 3}
  \end{subfigure}
  \caption{Ukázky jednoduchých sítí s grafem dosažitelnosti}\label{fig: ukázky jednoduchých sítí}
\end{figure}

\begin{itemize}
  \item 
  Síť na obrázku \ref{fig:ukázka grafu dosažitelnosti 1} \textbf{skončí} a je \textbf{slabě živá}.
  \item 
  Síť na obrázku \ref{fig:ukázka grafu dosažitelnosti 2} \textbf{skončí} a není \textbf{slabě živá}.
  \item 
  Síť na obrázku \ref{fig:ukázka grafu dosažitelnosti 3} je \textbf{vratná} a \textbf{živá}.
\end{itemize}

Všechny tři sítě jsou ohraničené 1. Také si můžeme všimnout že 
z výše uvedených vlastností u ukázkových sítí 
stačí jen vypsané vlastnosti a ostatní se dají odvodit ze závislosti vlastností.


\subsection{Graf pokrytí}

Hlavní nevýhodou grafu dosažitelnosti je, že může být
nekonečný a tudíž je nemožné ho zkonstruovat celý.
Můžeme velice jednoduše navrhnout a sestrojit triviální Petriho síť (Obrázek \ref{fig:neohraničená síť}) u které by konstrukce jejího grafu dosažitelnosti nikdy neskončila.

\begin{figure}[h]
  \centering
  \includegraphics[width=\linewidth]{net_unbounded_reachability}
  \caption{Příklad neohraničené sítě}\label{fig:neohraničená síť}
\end{figure}

Proto existuje upravená verze grafu dosažitelnosti nazvaný graf pokrytí,
který může obsahovat tzv. $\omega$ ohodnocení, které mimo celých 
čísel přiřadí alespoň jednomu místu i hodnotu $\omega$ 
symbolizující že místo může nabývat nekonečně vysokého počtu značek.
Petriho síť se nemůže nacházet v $\omega$ ohodnocení, toto ohodnocení je pouze
pro vytvoření abstrakce v grafu pokrytí.

Protože hodnotu $\omega$ bereme jako nekonečno pak od ní můžeme 
odečíst nebo přičíst libovolně velké číslo a hodnota se nezmění.
$$\dotsb = \omega - 2 = \omega - 1 = \omega = \omega + 1 = \omega + 2 = \dotsb$$
zároveň je taky vyšší jak libovolné přirozené číslo
$$ (\forall n \in \mathbb{N}_0)n < \omega $$

Ohodnocení $M$ značíme jako že je ostře menší $<$ než ohodnocení $M'$,
pokud pro každé místo $p$ platí $M(p) \leq M'(p)$ a alespoň pro jedno
místo $p$ platí $M(p) < M'(p)$.
$$
 M<M' = ((\forall p \in P) M(p) \leq M'(p)) \land (\exists p \in P) M(p) < M'(p)
$$

\subsubsection{Sestrojení grafu}

Sestrojování grafu probíhá postupně přidáváním hran. Nejdříve se přidá počáteční 
ohodnocení jako kořen grafu. Následně se z grafu vybírají náhodně
nevypočítané povolené přechody a pokud vedou do místa, které ještě 
v grafu není, tak se přidá a pokud je ostře menší než ohodnocení ze kterého
je dosažitelné, tak se přidají $\omega$ hodnoty na místa ve kterých má více značek.
Algoritmus končí výpočet až jsou všechny povolené 
přechody pro všechny vrcholy v grafu vypočítané.

\begin{center}
  Sestrojení grafu pokrytí pseudokód \ref{alg:MakeCoverabilityGraph} MakeCoverabilityGraph.
\end{center}

\begin{algorithm}
  \caption{MakeCoverabilityGraph}\label{alg:MakeCoverabilityGraph}
  \begin{algorithmic}[1]
    \Function{MakeCoverabilityGraph}{$\langle P,T,A,M_0\rangle$}
    \State $\langle V,E,v_0\rangle := \langle\{M_0\},\emptyset,M_0\rangle$
    \State $WorkSet := \emptyset $
    \ForAll{$t \in EnabledTransitions(P,T,A,M_0)$}
    \State $WorkSet := WorkSet \cup \{\langle M_0, t \rangle\} $
    \EndFor

    \While{$WorkSet \neq \emptyset$}
    \State $\langle M, t \rangle := RandomElement(WorkSet)$
    \State $WorkSet := WorkSet \setminus \{\langle M, t \rangle\}$
    \State $M' := FireTransition(P,t,A,M)$
    \ForAll{$\{M'' \in V \;|\; (M'' \to^* M \lor M'' = M) \land M'' < M'\}$}
    \ForAll{$p \in P$}
    \If{$M''(p)<M'(p)$}
    \State $M'(p) := \omega$
    \EndIf
    \EndFor
    \EndFor

    \If{$M' \notin V$}
    \State $V := V \cup \{M'\}$
    \ForAll{$t \in EnabledTransitions(P,T,A,M')$}
    \State $WorkSet := WorkSet \cup \{\langle M', t \rangle\} $
    \EndFor
    \EndIf
    \State $E := E \cup \{\langle M,t,M'\rangle\}$
    \EndWhile

    \State \textbf{return} $\langle V,E,v_0\rangle$
    \EndFunction
  \end{algorithmic}
\end{algorithm}

Pokud sestrojený graf pokrytí neobsahuje žádné $\omega$ ohodnocení,
pak je graf pokrytí totožný s grafem dosažitelnosti. 
Pokud graf pokrytí obsahuje $\omega$ ohodnocení, znamená to že 
graf dosažitelnosti by byl nekonečný a tudíž by nebylo možné 
ho zkonstruovat celý a nešli by na něm zjišťovat některé nebo všechny vlastnosti.
Proto si vystačíme s algoritmem na vytváření grafu pokrytí.

\subsubsection{Různé výsledky grafu pokrytí}

Při konstrukci grafu pokrytí záleží v jakém pořadí se hrany přidávají
a výsledný graf může mít různý počet vrcholů a hran 
v závislosti na pořadí přidávání hran.
V našem algoritmu využíváme funkci $RandomElement$, která vybere 
náhodný prvek z množiny a snažíme se tak tipovat, jaké pořadí hran bude
ideální pro sestrojení nejmenšího grafu. 
Pokud bychom chtěli sestrojit minimální 
graf pokrytí, museli bychom nahradit funkci $RandomElement$ nějakou
funkcí, která by vždy vybrala přechody právě 
v takovém pořadí, aby došlo k sestrojení minimálního grafu.

Že záleží na pořadí v jakém se hrany přidávají si můžeme ukázat na 
síti v obrázku \ref{fig:síť různé coverability}. Zde ale musíme 
dávat pozor, protože v tomto případě čísla neznačí jednotlivé přechody,
ale pořadí ve kterém byly hrany přidány.

\begin{figure}[h]
  \centering
  \includegraphics[width=\linewidth]{net_coverability_difference}
  \caption[Příklad sítě s rozdílnými grafy pokrytí. (fig.  14.1)]{Příklad sítě z knihy Understanding petri nets\cite{reisig2013understanding}(fig.  14.1) s rozdílnými grafy pokrytí. }\label{fig:síť různé coverability}
\end{figure}


\subsubsection{Upravená verze vlastností}\label{Upravená verze vlastností}

Oproti grafu dosažitelnosti náš graf pokrytí tak, jak jsme ho sestrojili 
pomocí algoritmu \ref{alg:MakeCoverabilityGraph} MakeCoverabilityGraph 
nemusí obsahovat všechny hrany přechodů, které mohou nastat a to znamená, 
že v některých případech některé vlastnosti Petriho sítě nejsme schopni určit, 
protože nám chybí informace o těchto chybějících hranách grafu pokrytí.
Problém je částečně způsobený tím, jak máme definovanou hodnotu $\omega$ a pokud 
nějaké místo $p$ je ohodnoceno $\omega$ pak už není možné, aby z vrcholu s tímto 
ohodnocením vedla hrana do vrcholu kde místo $p$ nebude mít hodnotu $\omega$.

\clearpage
\begin{figure}[h]
  \centering
  \includegraphics[width=0.8\linewidth]{net_undecidable_reversion}
  \caption{Příklad neohraničené sítě s chybějící hranou v grafu pokrytí}\label{fig:neohraničená síť reversible}
\end{figure}
V Obrázku \ref{fig:neohraničená síť reversible} vidíme červeně zvýrazněnou 
hranu, která při použití algoritmu \ref{alg:MakeCoverabilityGraph} MakeCoverabilityGraph
chybí. Přitom by tam měla hrana být, protože když budeme neustále opakovat 
aktivaci přechodu 2, tak se eventuálně (až bude počet aktivací přechodu 2 roven počtu aktivací 1) 
dostaneme do ohodnocení kde má místo $A$
hodnotu 0, což je zároveň výchozí ohodnocení. Díky této chybějící hraně 
bychom síť určili jako že není \textbf{vratná}, ale přitom ve skutečnosti je.
Proto musíme zjistit jestli jsou všechny vlastnosti grafu dosažitelnosti 
aplikovatlné i na graf pokrytí, případně poupravit nebo rozšířit jejich definici.
\\

Síť je \textbf{ohraničená}, pokud graf pokrytí neobsahuje žádné $\omega$ ohodnocení
\begin{proof}[Ohraničenost sítě v grafu pokrytí]
  \textbf{Ohraničenost} sítě je určená konečností grafu dosažitelnosti.
  Nekonečný rozvoj grafu dosažitelnosti je vždy v grafu pokrytí symbolizován
  $\omega$ ohodnoceními. Z toho můžeme vyvodit, že síť je 
  \textbf{ohraničená}, pokud její graf pokrytí neobsahuje žádné $\omega$ ohodnocení.
\end{proof}


Petriho síť ne\textbf{skončí}, pokud graf obsahuje $\omega$ ohodnocení.
\begin{proof}[Konečnost sítě v grafu pokrytí]
  Petriho síť ne\textbf{skončí}, pokud je její graf dosažitelnosti nekonečný tudíž,
  stejnou úvahou jako u ohraničenosti můžeme říct, že síť ne\textbf{skončí}
  pokud její graf pokrytí obsahuje $\omega$ ohodnocení a skončí, pokud je splněná
  původní podmínka v definici \ref{def:skončí}.
\end{proof}


Jestli je síť \textbf{vratná} z grafu s $\omega$ ohodnocením jsme si už ukázali 
že díky chybějícím hranám určitelné není. Můžeme si ale jednoduchou 
úvahou určit množinu případů kdy síť rozhodně vratná není, a to v 
situaci kdy existuje $\omega$ ohodnocení a neexistuje žádný přechod, 
který by měl větší vstup jak výstup.
\begin{proof}[Vratnost sítě v grafu pokrytí]
  Pokud máme $\omega$ ohodnocení tak to mimo jiné znamená že se suma značek všech 
  míst může při opakovaném aktivování
  některých přechodů zvyšovat neustále. Pak musí existovat i přechod,
  který tuto sumu snižuje neboli přechod, který má vyšší sumu
  násobků z místa $^\to t$ než do místa $t ^\to$. Pokud takový 
  přechod neexistuje přesto že v coverability grafu je $\omega$ ohodnocení,
  pak můžeme s jistotou říct že síť není \textbf{vratná}
\end{proof}
Pro určení, jestli je síť \textbf{vratná} v ostatních případech
bychom potřebovali algoritmus, který vytváří i zpětné hrany z $\omega$ ohodnocení.


Petriho síť je \textbf{bez mrtvého bodu}, pokud z každého vrcholu grafu 
pokrytí vede alespoň jedna hrana.
\begin{proof}[Mrtvý bod v grafu pokrytí]
  V tomto případě nemusíme rozlišovat $\omega$ ohodnocení a standardní ohodnocení,
  pokud z něj vede v grafu hrana, znamená to, že v tomto ohodnocení síť neuvázne.
\end{proof}

Petriho síť je \textbf{slabě živá}, pokud pro každý přechod existuje 
v grafu hrana označená tímto přechodem.
\begin{proof}[Slabě živá síť v grafu pokrytí]
  $\omega$ ohodnocení stejně jako v případě 
  určování jestli je síť \textbf{bez mrtvého bodu} 
  výsledek neovlivní. Pokud je přechod povolený, 
  nezáleží do jakého hodnocení vede, 
  proto nevadí když chybí jeho hrana do nižšího ohodnocení.
\end{proof}

Petriho síť je \textbf{živá}, pokud 
pro každý přechod $t$ a každé ohodnocení $M$ existuje v grafu cesta
z ohodnocení $M$ do ohodnocení ze kterého vede hrana z označením přechodu $t$.
\begin{proof}[Živá síť v grafu pokrytí]
 Když jsou ohodnocení $M < M'$, pak pokud je přechod $t$ 
  povolený v $M$ pak musí být povolený i v $M'$. Díky tomu můžeme určit
  že chybějící hrany z ostře větších ohodnocení do menších nejsou potřeba protože 
  by jejich existence stejně neumožňovala přístup k dalším přechodům a proto můžeme 
  živost vyčíst i z grafu pokrytí.
\end{proof}


\subsection{Příklady sítí}
Na obrázcích \ref{fig:síť kniha 1}, \ref{fig:síť kniha 2}, \ref{fig:síť kniha 3} jsou příklady sítí z knihy Understanding petri nets\cite{reisig2013understanding} 
s výsledky analýz editoru.
Odkaz na síť v knice je součástí popisů obrázků.

\clearpage
\begin{figure}[h]
  \centering
  \begin{subfigure}[h]{0.7\linewidth}
    \includegraphics[width=\linewidth]{net_cooking_vending_machine}
  \end{subfigure}
  \begin{subfigure}[h]{0.2\linewidth}
    \includegraphics[width=\linewidth]{net_cooking_vending_machine_analysis}
  \end{subfigure}
  \caption{Prodejní automat (fig 3.1)}
  \label{fig:síť kniha 1}
\end{figure}
\begin{figure}[h]
  \centering
  \begin{subfigure}[h]{0.7\linewidth}
    \includegraphics[width=\linewidth]{net_mutual_exclusion}
  \end{subfigure}
  \begin{subfigure}[h]{0.2\linewidth}
    \includegraphics[width=\linewidth]{net_mutual_exclusion_analysis}
  \end{subfigure}
  \caption{Vzájemné vyloučení (fig 3.2)}
  \label{fig:síť kniha 2}
\end{figure}
\begin{figure}[h]
  \centering
  \begin{subfigure}[h]{0.6\linewidth}
    \includegraphics[width=\linewidth]{net_crosstalk}
  \end{subfigure}
  \begin{subfigure}[h]{0.2\linewidth}
    \includegraphics[width=\linewidth]{net_crosstalk_analysis}
  \end{subfigure}
  \caption{Crosstalk algoritmus (fig 3.7)}
  \label{fig:síť kniha 3}
\end{figure}



\clearpage
\section{Editor}
\subsection{Systémové požadavky}
\begin{itemize}
  \item Operační systém: Windows 10 (starší verze windows netestovány)
  \item Ovládání: klávesnice + myš
  \item Rozlišení obrazovky: minimálně 720p
\end{itemize}

  
\subsection{Rozložení editoru}
Editor je rozložený na několik částí. Všechny tyto části 
jsou písmeny označené v
obrázku \ref{fig:Rozložení editoru} 
Rozložení editoru. Každá část editoru je popsaná ve své vlastní sekci.

\begin{figure}[h]
  \centering
  \begin{subfigure}[h]{350px}
    \includegraphics[height=300px, width=350px]{full_image_splited}
  \end{subfigure}
  \caption{Rozložení editoru}
  \label{fig:Rozložení editoru}
  \begin{subfigure}[h]{0.55\textwidth}
    \begin{tabular}{|c l c|}
      \hline
      označení &      název části editoru &    sekce \\
      \hline
      \hline
      P &             Postranní panel&  \ref{panel} \\
      HP &            Hlavní plocha editoru&  \ref{hlavní plocha} \\
      B &             Panel nástrojů editoru &  \ref{panel nástrojů} \\
      T &             Tabulka ohodnocení &  \ref{tabulka ohodnocení} \\
      A &             Výsledky analýzy &  \ref{výsledky analýzy} \\
      \hline
    \end{tabular}
  \end{subfigure}
\end{figure}

\subsubsection{Postranní panel}\label{panel}

Postranní panel obsahuje tlačítka pro práci se záložkami a 
samotné záložky. Pod označením \textbf{FILE} jsou tlačítka:
\begin{center}
  \begin{tabular}{c p{0.8\linewidth}}
    \textbf{New}    & Vytvoří novou prázdnou záložku. \\
    \textbf{Load}   & Otevře dialog pro načtení uložené sítě. \\
    \textbf{Save}   & Uloží síť v otevřené záložce. Pokud byla síť již uložena 
      nebo načtena, dialog nabídne uživateli na výběr dvě možnosti.
      Možnost \textit{yes} uloží a přepíše původní soubor.
      Možnost \textit{Select file} otevře dialog 
      a uživatel vybere vlastní místo uložení. \\
    \textbf{Close}  & Zavře aktuálně otevřenou záložku. \\
  \end{tabular}
\end{center}

Pod označením \textbf{Tabs} jsou pak záložky které se otevírají kliknutím.
Záložky jsou nazvané podle jména souboru sítě.

\begin{figure}[h]
  \centering
  \includegraphics{editor_panel}
  \caption{\\Postranní panel}\label{fig:Postranní panel}
\end{figure}

\subsubsection{Ovládání, Hlavní plocha editoru}\label{hlavní plocha}

Editor je navržený tak aby bylo možné jej používat pouze za použití 
myši bez využití klávesnice. Zároveň oproti ostatním editorům
nemá různé nástroje (např. na vkládání různých objektů) a všechny 
akce editování jsou možné bez toho, aby kurzor opustil hlavní plochu editoru.

Kliknutí levým tlačítkem myši vytvoří přechod. 
Kliknutím na přechod se začne vytvářet hrana. Pokud se při 
vytváření hrany klikne do nějakého místa, tak se na něj hrana 
připojí. Pokud se klikne do prázdného prostoru, vytvoří se zde nové místo.
Vytváření hrany jde zrušit pravým tlačítkem myši.

Při kliknutí na hranu nebo místo se otevře dialog 
na obrázku \ref{fig:editace hodnot} editace hodnoty.
Při najetí nad editované pole se zobrazí šipky, kterými 
je možné přidávat nebo ubírat hodnotu, nebo je možné taky 
do něj kliknout, aby se zde umístil kurzor klávesnice a 
přitom, když je myš stále nad polem použít kolečko na změnu hodnoty.
Dialogy se ukládají kliknutím na OK nebo stisknutím Enter. 
Kliknutí(levé i pravé) do jíného prostoru
hlavní plochy způsobí zavření dialogu bez provedení změny.

Ohodnocení místa se dá měnit i bez otevření dialogu a to najetím myši
a rolováním kolečka, zatímco při najetí a rolovaní na hranu pouze prohodí směr
hrany. Rolovaní při najetí na přechody pouze odebere nebo přidá přechodu 
$\epsilon$ označení(označení pouze orientační pro uživatele).

Kliknutím a tažením je pak možné jednotlivá místa a přechody přesouvat.
Při přesouvání se jednotlivé objekty navzájem odpuzují aby nedošlo k jejich překrytí.

Nakonec pravým tlačítkem je možné místo, přechod nebo hranu odstranit.

Při editaci je možné také využít klávesových zkratek které jsou popsané v sekci \ref{zkratky}.

Pokud uživatel nevyžaduje zobrazené analýzy a tabulku ohodnocení, je možné je tlačítkem nad nimi skrýt.

\begin{figure}[h]
  \centering
  \begin{subfigure}[h]{0,4\linewidth}
    \includegraphics{dialog_place}
    \caption{Editace ohodnocení místa}
  \end{subfigure}
  \begin{subfigure}[h]{0,4\linewidth}
    \includegraphics[width=\linewidth]{dialog_arc}
    \caption{Editace násobků hrany}
  \end{subfigure}
  \caption{Editace hodnot}
  \label{fig:editace hodnot}
\end{figure}

Pokud je mód nastavený na hodnotu \textit{Run} výše popsané akce myší 
se vypnou. Jediná akce v tomto módu je aktivace přechodu.

\clearpage
\subsubsection{Panel nástrojů editoru}\label{panel nástrojů}

Funkce tlačítek panelu nástrojů zobrazeného v obrázku \ref{fig:Panel nástrojů editoru}:
\begin{center}
  \begin{tabular}{r p{0.6\linewidth}}
    \textbf{Zpět/Vpřed}          & Vrátí poslední akci nebo zruší vrácení poslední akce. \\
    \textbf{Tisk}                & Otevře dialog pro tisk otevřené sítě. \\
    \textbf{Přiblížit/Oddálit}   & Přiblíží/Oddálí Petriho síť. \\
    \textbf{Skrýt/Zobrazit popisky}  & Skryje nebo zobrazí popisky míst a přechodů. \\
    \textbf{Změna módu}  & Přepíná mezi \textbf{editovacím módem} a módem \textbf{spouštění sítě}. \\
  \end{tabular}
\end{center}


\begin{figure}[h]
  \centering
  \includegraphics[width=\linewidth]{bar}
  \caption{Panel nástrojů editoru}\label{fig:Panel nástrojů editoru}
\end{figure}

\subsubsubsection{Tisk}\label{tisk}

Hlavní plocha má vždy rozměry pro tisk na papír formátu a4. Pro tisk 
do PDF je potřeba virtuální tiskárna, která tuto funkci umožňuje (např. Microsoft print to PDF).

Tisk nemusí vždy fungovat, je to způsobené chybou ve verzi Electronu 
která byla k vytvoření programu použita. Pokud tisk 
nefunguje mělo by stačit editor restartovat.

\subsubsection{Tabulka ohodnocení}\label{tabulka ohodnocení}

Tabulka ohodnocení (Obrázek \ref{fig:Tabulka ohodnocení obrázek}) slouží pro rychlé testování sítě uživatelem.
Každý řádek v tabulce reprezentuje jedno dosažitelné ohodnocení 
v levé části tabulky a povolené přechody v tomto ohodnocení jsou 
zobrazeny zelenými obdelníčky v pravé části tabulky.
Uživatel může kdykoliv kliknout na jakýkoliv ze zelených obdelníčků 
a tím přidá další řádek pod řádek na který kliknul, který bude osahovat 
nové ohodnocení vzniklé aplikací přechodu který uživatel vybral.
Pokud uživatel vybral přechod na řádku pod kterým už řádky jsou, 
tak budou všechny odebrány a pak až se přidá nový.

Pokud uživatel najede na ikonu oka dojde k zobrazení ohodnocení 
daného řádku do sítě na hlavní ploše editoru.

\begin{figure}[h]
  \centering
  \includegraphics[width=\linewidth]{tabulka}
  \caption{Tabulka ohodnocení}\label{fig:Tabulka ohodnocení obrázek}
\end{figure}

\subsubsection{Výsledky analýzy}\label{výsledky analýzy}

Ve výsledcích analýzy (viz Obrázek \ref{analýza obrázek}) jsou zobrazeny vlastnosti sítě podle sekce \ref{Upravená verze vlastností}.
Jak jednotlivé vlastnosti přečíst je popsané v tabulce \ref{vlastnosti site v editoru}.

\begin{table}[h]
  \centering
  \begin{tabular}{| c | c | r l |}
  \hline
    Název v editoru & Vlastnost & Hodnota & Význam \\
    \hline
    \hline
     states & - & číslo & počet stavů sítě \\ 
    \hline
    \multirow{2}{*}{bounded} & \multirow{2}{*}{\textbf{ohraničenost}} & \textit{$\omega$} &  síť je \textbf{neohraničená} \\ 
    &                   & číslo & určuje \textbf{ohraničenost} \\ 
    \hline
\multirow{2}{*}{terminates} & \multirow{2}{*}{síť \textbf{skončí}} & \textit{Yes} & síť \textbf{skončí} \\ 
    &                   & \textit{No} & síť ne\textbf{skončí} \\ 
    \hline
\multirow{3}{*}{reversible} & \multirow{3}{*}{\textbf{vratná}} & \textit{Yes} & síť je \textbf{vratná} \\ 
    &                   &   \textit{No} & síť není \textbf{vratná}  \\
    &                   & \textit{?} & nevíme zda je síť \textbf{vratná} \\ 
    \hline
\multirow{2}{*}{deadlock-free} & \multirow{2}{*}{\textbf{mrtvý bodu}} & \textit{Yes} & síť je \textbf{bez mrtvého bodu} \\ 
    &                   &  \textit{No} & síť má \textbf{mrtvý bod}  \\ 
    \hline
\multirow{3}{*}{live} & \multirow{3}{*}{\textbf{živá} / \textbf{slabě živá}} 
                         & \textit{Yes} & Síť je \textbf{živá} \\ 
    &                   & \textit{Weakly}  & Síť je \textbf{slabě živá} \\ 
    &                   & \textit{No} & Síť není \textbf{slabě živá} \\ 
    \hline

  \end{tabular}
  \caption{Určování vlastností sítě v editoru}\label{vlastnosti site v editoru}  
\end{table}

\clearpage

\begin{figure}[h]
  \centering
  \includegraphics{analysis}
  \caption{Ukázka výsledků analýzy sítě}\label{analýza obrázek}
\end{figure}

\subsubsection{Klávesové zkratky}\label{zkratky}

Seznam klávesových zkratek použitelných v editoru je v tabulce \ref{tabulka zkratky}.
\begin{table}[h!]
  \centering
  \begin{tabular}{| c | l |}
    \hline
    Zkratka & Akce \\
    \hline
    Ctrl + Z      & Zpět \\
    Ctrl + Shift + Z & Vpřed \\
    Ctrl + O & Otevření uložené sítě \\
    Ctrl + S & Uložení sítě na aktuální záložce\\
    Ctrl + N & Vytvoření nové sítě \\
    \hline
  \end{tabular}
  \caption{Klávesové zkratky využitelné v editoru}\label{tabulka zkratky}
\end{table}




\section{Použité technologie}

Následující technologie byly použity pro vytvoření editoru Petriho sítí.


\subsection{NodeJS}
Technologie která umožňuje využívat jazyk JavaScript pro psaní 
serverových aplikací. Cílem platformy NodeJS je vytvořit
ekosystém pro jednoduší vývoj webových stránek a aplikací, 
kde stačí pro vytváření funkcionality pouze jeden programovací jazyk.


\subsection{TypeScript}
TypeScript je opensource programovací jazyk od společnosti Microsoft, který je nádstavbou nad jazykem JavaScript.
Jelikož je TypeScript nádstavbou nad programovacím jazykem JavaScript, tak je jakýkoliv validní kód v JavaScriptu validním kódem v TypeScriptu.
TypeScript se kompiluje do Javascriptu a proto po stránce 
funkcionality nenabízí nic navíc, avšak po stránce vývoje 
nabízí možnost statické typové kontroly
se kterou je spjaté fungování našeptávačů v dnešních textových 
editorech a také nabízí možnost kompilace do 
starších verzí JavaScript se simulací funkcionality 
novějších verzí JavaScriptu.

Zdrojový kód \ref{TypeScript} je ukázka jednoduchého kódu v TypeScriptu. Můžeme
si zde všimnou proměnné \textit{a} do které nakonec nahrajeme 
řetězec, i přesto že 
ji máme definovanou jako číslo. Proto si musíme zároveň sami dávat pozor na to jestli
tento jazyk využíváme správně a neobcházíme ho, tak že bychom si jím spíš škodili než pomohli.
\begin{kicode}{JavaScript}{TypeScript}{Ukázka TypeScriptu}
  // typová inference, a je číslo
  let a = 10;
  // řetězec není číslo CHYBA! - nezkompiluje se
  a = "necislo";
  // 12 je číslo, validní přiřazení
  a = 12; 
  // obejití typovosti, proběhne v pořádku ale POZOR 
  // v a už není číslo, ikdyž podle TypeScriptu to tak vypadá
  (a as any) = "necislo"
  // definice typu s inicializací, b je číslo nebo text
  let b: number | string = 10;
  /**
   * Funkce fnc je funkce jednoho argumentu který 
   * je složený objekt obsahující
   * vlasnosti x a y které jsou čísla.
   * Funkce fnc vrací pravdivostní hodnotu.
   */
  function fnc(arg: {x:number,y:number}): boolean {
      // Kód funkce, pokud všechny větve funkce 
      // nevrací definovanou návratovou hodnotu vyhodí chybu
  }

\end{kicode}


\subsection{Electron}
Electron je opensource framework vytvořený v NodeJS, který 
zaobaluje Windows API (viz zdrojový kód \ref{electron vyrvoření okna}) a dohromady se softwarem chromium umožňují 
vytváření okenních aplikací za pomocí technologií HTML, CSS a JavaScript nebo TypeScript.
Electron je stejně jako webová stránka nebo webová aplikace rozdělený
na dvě části. \uv{Serverová} část komunikuje s Windows API a 
\uv{klientská} tzv. renderer část se stará už pouze o vykreslování 
okna. Obě části jsou pak propojené přes \textit{Inter-Process Communication}, 
což uživatel může využívat přes objekty \textit{ipcRenderer} na straně rendereru
a přes \textit{ipcMain} na straně serveru (viz Zdrojový kód \ref{electron komunikace}).

\begin{kicode}{JavaScript}{electron vyrvoření okna}{Vytváření okna v elektronu}
  import { BrowserWindow, Menu, app } from "electron";
  function createWindow() {
      // vytvoření samotného okna
      const mainWindow = new BrowserWindow({
          width: 400, height: 500,
          title: 'NazevOkna',
      });
      // otevře výchozí stránku
      mainWindow.loadURL(`file://${__dirname}/index.html`);
      // odebere horní menu aplikace
      Menu.setApplicationMenu(null);
  }
  
  // počká až vše bude připraveno
  app.on('ready', createWindow);
\end{kicode}

\begin{kicode}{JavaScript}{electron komunikace}{Komunikace v rámci procesu elektronu}
  //server
  ipcMain.on("user-event-name", (e, arg)=>{
    // Do Something
  })

  //renderer
  ipcRenderer.send("user-event-name", {/*DATA*/})
\end{kicode}


\subsection{Javascriptová Knihovna Data driven documents (D3)}
Knihovna D3 se používá pro zobrazení dat do 
\textit{document object modelu}.
Knihovna obsahuje selektory, které umožňují vybrat zároveň data 
i DOM objekty a pracovat s nimi najednou. Hlavní výhodou knihovny 
je její rozdělení selektorů na \textbf{Enter}, \textbf{Exit}, \textbf{Update}.
Selektor \textbf{Enter} se volá, pokud dojde k případu, že má kolekce 
dat větší počet prvků než je vytvořeno DOM elementů do kterých se mají zobrazovat.
\textbf{Enter} je tedy selektor který definuje jak se vytváří nové DOM elementy 
když přibydou data. Naopak \textbf{Exit} selektor je pravým opakem \textbf{Enter}
selektoru a stará se tedy o případy kdy máme méně dat než máme DOM elementů.
Poslední selektor je \textbf{Update}, který je standardním selektorem 
(jako je například přímo v JavaScriptu document.querySelector), který přímo 
aplikuje změny dat do DOM elemntů. Selektory \textbf{Enter} a \textbf{Exit} 
mají své příkazy, zatímco selektor \textbf{Update} speciální příkaz nemá a je to jen 
souhrnné označení všech ostatních selektorů. V ukázce (viz Zdrojový kód \ref{D3}) je
jednoduchý kód, který pro každý prvek dat zobrazí jeden řádek s těmito daty.

\begin{kicode}{JavaScript}{D3}{Ukázka kódu v knihovně D3}
import * as d3 from 'd3';

// vybere v DOM element co má id container
const container = d3.select("#container");
// data která chceme zobrazit 
const data = [1, 2, 3, 4];
// propojíme data s DOMem
container.data(data)
    // enter selektor
    .enter()
    // pro data které nemají element jej přidáme
    .append("li")
    // přejde zpět na update selektor
    .merge(container)
    // změní text li elementu na číslo z dat
    .text(d => d)
    // exit selektor
    .exit()
    // odebere přebívající DOM elementy
    .remove()
    ;
  
\end{kicode}


\subsection{Scalable vector graphics (SVG)}
SVG je značkovací jazyk vycházející z XML, který popisuje vykreslovaní 
vektorové grafiky. Je ideální pro vytváření grafiky, kde je vyžadováno
aby se dala přibližovat a přitom se neztrácela kvalita obrazu (proto Scalable).
V editoru je tato technologie využita v samotnému vykreslovaní sítí.


\subsection{JavaScript object notation (JSON)}
JSON je technologie která využívá notace javascriptových objektů pro ukladání dat.
Výhodou tohoto formátu je, že po rozparsování souboru se může s ním hned v 
javascriptu jednoduše pracovat a také je čitelný pro člověka a je podporovaný ve většině softwaru.
Sítě vytvořené v editoru jsou ukládané v tomto formátu.
Uložené sítě nejsou určené aby je někdo editoval, ale uložená síť může být třeba 
použita v jiném programu, který si ji rozparsuje a bude s ní dál pracovat.
V ukázce (viz Zdrojový kód \ref{Uložení sítě}) je síť tvořená jedním místem s ohodnocením 2, 
jedním přechodem a hranou směřující do místa.
\clearpage

\begin{kicode}{JavaScript}{Uložení sítě}{Uložení Petriho sítě}
{
  "places": [
    {
      "id": 0,
      "position": {
        "x": 95, "y": 35
      },
      "marking": 2
    }
  ],
  "transitions": [
    {
      "position": {
        "x": 30, "y": 35
      },
      "id": 0,
      "isCold": false
    }
  ],
  "arcs": [
    {
      "place_id": 0,
      "transition_id": 0,
      "toPlace": 1,
      "toTransition": 0
    }
  ]
}
\end{kicode}




\section{Stavba programu}

Editor je napsán hlavně v renderer straně elektronu. Díky tomu je možné 
editor v případě potřeby v budoucnu předělat na webovou aplikaci.
Struktura programu (Obrázek \ref{Struktura programu}) je tvořená 
souborem \textit{program.ts}, který má na starosti vytváření hlavního okna 
a načíst/uložit dialogů. Soubor \textit{main.ts} obsluhuje vytváření
a ukládání editorů. Složky CORE a Helpers obsahují univerzální 
funkcionalitu používanou editorem a složka Editor obsahuje pak 
samotný editor tvořený třídami Model, Draw a 
třídami které je propojují do jednoho celku.


\begin{figure}[h]
  \centering
  \includegraphics[scale=1]{struktura_programu}
  \caption{Struktura programu}\label{Struktura programu}
\end{figure}


\subsection{Modely}

V programu jsou třídy nazvané model, které obsahují data a 
starají se o jejich zpracování.
Tyto modely obsahují také definice pro 
převod mezi anonymním objektem a objektem 
vytvořeným z těchto tříd a díky tomu je 
jednoduché je serializovat a deserializovat.


\subsection{Draw třídy}

Draw třídy obsahují logiku k zobrazovaní dat obsažených v modelech a taky
zároveň odchytávání akcí způsobených na DOM elementech reprezentujících 
data v modelech.


\subsection{Třída editor}

Třída editor propojuje všechny potřebné třídy a zároveň obsahuje obsluhu 
akcí provedených na třídách draw. Každá síť při načtení nebo vytvoření 
má vlastní instanci třídy editor.




\section{Kompilace}

Kompilace programu vyžaduje nainstalovaný \textit{NodeJS}. Pro kompilaci pak
stačí pouze spustit skript \textit{src\char`\\Build.bat} a program se zkompiluje do složky 
\textit{src\char`\\PNetEditor\char`\\dist\char`\\} kde je jako spustitelný (.exe) soubor.

Pokud dojde ke změnám v TypeScriptových(.ts) souborech, je třeba zkompilovat i je.
Ke kompilaci TypeScriptových souborů je potřeba mít visual studio 2017(instalační soubor na CD)
s NodeJS sadou nástrojů. Stačí jen otevřít \\
\textit{PNetEditor.sln} a v horní liště 
visual studia a vybrat možnost \textit{Build$\to$Rebuild solution}.




\section{Srovnání s ostatními editory}

Ostatní editory nabízí více funkcionality například ve směru analýz, 
ukládání a tisku. Většina editorů vytváří jednotlivé prvky sítě tak,
že na každý prvek existuje nástroj v nástrojové liště. Oproti tomu 
tento editor nevyžaduje na vytváření sítě žádné přepínání nástrojů.
Na základě Fittova zákonu, když není potřeba přejíždět pořád nahorů do lišty 
pro změnu nástrojů a díky tomu je tento editor rychlejší na vytváření sítí.
Tento editor také vyčnívá svou tabulkou ohodnocení, kde si uživatel velice 
rychle může vyzkoušet a na jednom místě vidět, jestli síť splňuje jeho požadavky.




\section{Obsah přiloženého CD} \label{sec:ObsahCD}

\begin{description}
  \item[\texttt{bin/}] \hfill \\
        Adresář obsahuje soubor \textit{pnet\_editor.exe} který program spustí. 

  \item[\texttt{doc/}] \hfill \\
        Text práce ve formátu PDF, vytvořený s~použitím závazného stylu KI
        PřF UP v~Olomouci pro závěrečné práce, včetně všech příloh,
        a~všechny soubory potřebné pro bezproblémové vygenerování PDF
        dokumentu textu (v~ZIP archivu), tj.~zdrojový text textu, vložené
        obrázky, apod.

  \item[\texttt{src/}] \hfill \\
        Kompletní zdrojové texty programu \textsc{Editor Petriho sítí} 
        se všemi potřebnými (příp.~převzatými) zdrojovými
        texty, knihovnami a~dalšími soubory potřebnými pro bezproblémové
        vytvoření spustitelných verzí programu. Ve složce se nachází
        soubor řešení Visual Studia 2017 \textit{PNetEditor.sln},
        soubor na spouštění kompilace \textit{Build.bat} a
        složka s kódy a knihovnami.
        

  \item[\texttt{readme.txt}] \hfill \\
        Tento soubor obsahuje informace pro spuštění a kompilaci bakalářské práce 

\end{description}

Navíc CD/DVD obsahuje:

\begin{description}

  \item[\texttt{examples/}] \hfill \\
        Obsahuje uložené příklady sítí ze sekce \ref{příklady sítí}.

  \item[\texttt{install/}] \hfill \\
        Instalátory aplikací, knihoven potřebných pro kompilaci programu.

\end{description}

U~veškerých cizích převzatých materiálů obsažených na CD/DVD jejich
zahrnutí dovolují podmínky pro jejich šíření nebo přiložený souhlas
držitele copyrightu. Pro všechny použité (a~citované) materiály,
u~kterých toto není splněno a~nejsou tak obsaženy na CD/DVD, je uveden
jejich zdroj (např.~webová adresa) v~bibliografii nebo textu práce
nebo v souboru \texttt{readme.txt}.


\begin{kiconclusions}
  Při vytváření tohoto editoru jsem se snažil najít něco, čím bude 
  editor vyčnívat od ostatních editorů. Proto jsem se rozhodl zaměřit
  se na rychlost vytváření sítě za použití pouze myši. Myslím si, že se 
  mi povedlo vytvořit editor, ve kterém se s trochou cviku dá vytvořit 
  libovolná síť velice rychle a pohodlně.

  Díky této práci jsem posunul své znalosti dál. Vyzkoušel jsem si různé 
  způsoby vytváření GUI a také se obeznámil s touto problematikou jako celkem.
  Také jsem rád že jsem díky této práci o trochu obohatil své znalosti 
  na poli teoretické informatiky.
\end{kiconclusions}

\begin{kiconclusions}[english]
  When creating this editor, I tried to find something in which would be
  editor stand out from other. That's why I decided to focus
  at the speed of making Petri nets using only mouse. I think that
  I managed to create an editor in which I can create with a little practice
  arbitrary network very quickly and conveniently.

  Thanks to this work I have moved my knowledge further. I tried different
  ways to create GUIs and also become familiar with this issue as a whole.
  I am also glad that I have enriched my knowledge with this work
  in the field of theoretical computer science.
\end{kiconclusions}

\nocite{*}

\printbibliography



\end{document}